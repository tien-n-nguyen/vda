\subsubsection{\bf Comparison on Vulnerability Assessment on C Dataset (RQ2)}




\begin{table}[t]
	\caption{Vulnerability Assessment on C Dataset (RQ2)}
        \vspace{-9pt}
	\begin{center}
%		\footnotesize
                \small
		\renewcommand{\arraystretch}{1}
		\begin{tabular}{l|p{1.9cm}<{\centering}|p{1.5cm}<{\centering}|p{1.5cm}<{\centering}}
			\hline
			\multirow{2}{*}{CVSS Metric}     & \multirow{2}{*}{Evaluation Metric}  & \multicolumn{2}{c}{Model}\\
			\cline{3-4}
		                                     &                                     & DeepCVA~\cite{deepCVA-ase21}    & {\tool}       \\
		    \hline
			\multirow{2}{*}{Confidentiality} & macro F1-score                             &     0.50       & 0.65\\
			\cline{2-4}
			                                 & MCC                                 &      0.23      & 0.31\\
			\hline
			\multirow{2}{*}{Integrity}       & macro F1-score                             &    0.42        & 0.55\\
			\cline{2-4}
			                                 & MCC                                 &    0.24        & 0.33\\
			\hline
			\multirow{2}{*}{Availability}    & macro F1-score                             &   0.47         & 0.63\\
			\cline{2-4}
			                                 & MCC                                 &    0.28        & 0.34\\
			\hline
			\multirow{2}{*}{Access Vector}   & macro F1-score                             &   0.58         & 0.69\\
			\cline{2-4}
			                                 & MCC                                 &    0.22        & 0.31\\
			\hline
			\multirow{2}{*}{Access Complexity} & macro F1-score                           &   0.49         & 0.66\\
			\cline{2-4}
			                                 & MCC                                 &    0.26        & 0.35\\
			\hline
			\multirow{2}{*}{Authentication}  & macro F1-score                             &   0.67         & 0.72\\
			\cline{2-4}
			                                 & MCC                                 &   0.36         & 0.39\\
			\hline
			\multirow{2}{*}{Severity}        & macro F1-score                             &   0.44         & 0.58\\
			\cline{2-4}
			                                 & MCC                                 &   0.23         & 0.28\\
			\hline
			\hline
			\multirow{2}{*}{Average}         & macro F1-score                             &    0.51        & 0.64 ($\Uparrow${\bf 25.5\%})\\
			\cline{2-4}
			                                 & MCC                                 & 0.20           & 0.33 ($\Uparrow${\bf 26.9\%})\\
            \hline
		\end{tabular}
		\label{rq1_results}
	\end{center}
\end{table}

In Table~\ref{rq1_results}, {\tool} relatively improves over DeepCVA~\cite{deepCVA-ase21} by {\em 25.5\% in macro F1-score and
26.9\% in multi-class MCC} on the overall multi-class classification. For individual VATs, {\tool} improves upon DeepCVA by {\em 7.5--34.7\% in macro-F1-score and 8.3--40.9\% in multi-class MCC}. We can see that {\tool} consistently outperforms DeepCVA on all VAT types, thus corroborating with the design choices in Key Ideas 1--3 (see Section~\ref{key-ideas:sec}). Moreover, we can see that the largest relative improvement in macro F1-score and multi-class MCC happens for {\em Access Complexity} and {\em Access Vector}, respectively. Such gains in performance can possibly be attributed to the nature of Access VAT types, the access information for which, is more often than not, extensively checked in the changed code context, which is well represented in {\tool} but not in DeepCVA.
%Moreover, the largest relative improvement in macro F1-score happens for {\em Access Complexity} and the largest relative improvement in multi-class MCC happens for {\em Access Vector}. The lowest relative improvement in both macro F1-score and multi-class MCC happens for type {\em Authentication}, however, the absolute macro F1-score (0.72) for {\em Authentication} is highest among all the VATs.



%Table~\ref{rq1_results} shows that {\tool} outperforms the state-of-the-art baseline, DeepCVA, in the automation of commit-level vulnerability assessment on the C dataset. Particularly, {\tool} improves the state-of-the-art baseline by 25.5\% in terms of macro-F1-score, and 26.9\% in terms of multi-class MCC on the overall performance. The higher values of macro-F1-score and MCC indicate that {\tool} can perform more accurate multi-classification on C commits than DeepCVA. As for specific types of vulnerability assessment, {\tool} improves the state-of-the-art baseline by 7.5-34.7\% in terms of macro-F1-score and 8.3-40.9\% in terms of macro-F1-score. The lowest relative improvement of both macro-F1-score and multi-class MCC happens when testing the $Authentication$. At the same time, the biggest relative improvement of macro-F1-score happens when testing $Access Complexity$ and the biggest relative improvement of multi-class MCC happens when testing $Access Vector$.























