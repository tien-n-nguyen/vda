\section{Introduction}
\label{intro:sec}

%The need for cyber resilience is increasingly important in our~techno\-logy-dependent society, where software systems have been, and will continue to be the target of cyber attackers. 
Software Vulnerabilities (SVs) are security weaknesses and flaws that
are exploitable in the cyber-attacks. New software security
vulnerabilities are discovered on an almost daily basis. It is crucial
to identify software vulnerabilities as early as possible, because
late corrections of errors could cost much more than early correction
or even cause more severe damage. Recognizing this, researchers
propose the approaches to detect software vulnerabilities as soon as
new code changes are committed to the code repositories during
software development. Those approaches are referred to as {\em
  commit-level vulnerability detection
  (VD)}~\cite{perl2015vccfinder,zhou2017automated,chen2019large}. They
could be used to warn developers early on potential vulnerable code.

For potential vulnerable code, it would be also useful to provide
developers with the assessment on the impacts of the vulnerability on
the system under development. In the Common Vulnerability Scoring
System (CVSS)~\cite{first-website}, the security experts and analysts
have manually provided the assessments in terms of numerical ratings
to quantify different aspects of a software vulnerability including
the exploitability, the impacts and the severity of the attacks, the
level of damages of the vulnerabilities,
etc. (Figure~\ref{CVSS-tab}). The numerical scores can be transformed
into qualitative representations (such as low, medium, high, and
critical) to help organizations properly assess and prioritize their
vulnerability management processes~\cite{first-website}. Despite its
usefulness, there are restrictions. First, human efforts are needed
for the {\em manual assessment}, leading to the delays in the process.
Second, CVSS scores are provided {\em after} a vulnerability was
reported. Thus, for an early action, an {\em automated, vulnerability
  assessment} is desired at the committing time. The {\em
  vulnerability detection and assessment tools (VDA)} can be
integrated into the repositories as the code is checked in to provide
just-in-time assistance.

%Le {\em et al.}~\cite{deepCVA-ase21} proposed DeepCVA,

Recognizing such importance, the state-of-the-art, commit-level
software vulnerability assessment tools~\cite{deepCVA-ase21} have been
proposed to assess a committed code change. Reseachers have leveraged
the aforementioned CVSS ratings manually assessed by security analysts
for the vulnerabilities as labeled data to train a machine learning
(ML) model and applied it on the code change to predict the assessment
ratings. However, none of the existing commit-level vulnerability
assessment tools supports {\em both commit-level vulnerability
  detection and assessment}. Importantly, they are still limited in
the {\em representation of code changes} and do not take into account
the {\em context} of such changes.

%Despite being first to tackle commit-level SV assessment,
%DeepCVA~\cite{deepCVA-ase21} still suffers low~accuracy.  In general,
%it has three main drawbacks. Specifically, there is a limitation on
%{\bf the representation of code changes} and surrounding code.
Specifically, they are limited in using $n$-grams in representing code
changes. $n$-grams are insufficient to capture the execution flows
among code elements surrounding the changes. {\em The important
  statements having dependencies with the changed statements can be
  useful for the assessment, but can be far apart}. For example, many
Denial of Service (DoS) vulnerabilities are related to null-pointer
exceptions, exception flows, segmentation faults, etc. The execution
flows among statements are not well-represented via $n$-grams with
such limited lengths. Moreover, {\em the statements in an $n$-gram
  might be irrelevant to the current vulnerability}. Thus, the
assessment of a DoS vulnerability with $n$-grams could be
imprecise. {\em $n$-grams also enforce an order in source code}, which
might not be the execution order, e.g., in the cases of loop,
condition, or recursion. The execution order and dependencies among
the statements are important in an DoS. Thus, to assess the impact of
%a DoS, or a general impact of
a vulnerability, $n$-gram is limited.

Second, to estimate the impacts of software vulnerability, a model
needs to consider {\bf program dependencies} among statements since an
attack to a vulnerability involves the~exploits of the control and
data dependencies. The state-of-the-art vulnerability assessment tools
capture code changes as code tokens in short sequences with $n$-grams,
thus, does not model the program dependencies. Finally, it
does not represent well the {\bf code context} surrounding a
change. The same change occurring in different contexts might cause
different effects, leading to different impact grades. In the
state-of-the-art approaches~\cite{deepCVA-ase21}, the context of
un-changed code is not~distinguishable and not represented
separately from the code changes. The model can be confused by the two
training instances having the same combination of code changes and
contexts, but with different code changes and contexts
themselves. Those limitations lead to low accuracy in the
existing approaches.

%Technical paras

We present {\tool}, a Context-aware, Graph-based, Commit-level
Vulnerability Detection and Assessment Model to evaluate a changed
code, detect any vulnerability and provide the CVSS
assessment grades for the detected vulnerability. To build {\tool}, we
provide a novel integration among three key ideas. First, to make
prediction on the CVSS assessment grades on code changes, we develop a
novel {\bf context-aware, graph-based, representation learning 
  model} to {\bf learn the contextualized embeddings for the code
  changes} that integrate {\em program dependencies}, and the
surrounding {\em contexts} of code changes.  Both versions of the
program dependence graphs (PDGs) before and after the commit are
modeled via the multi-version {\mvpdg}~\cite{flexeme-fse20}.
%built from the PDGs before and after the changes.
%{\mvpdg} is a directed graph generated from the disjoint union of all
%nodes and edges in the PDGs at the versions before and after the
%changes.
To build such embeddings for code changes, {\tool} {\em explicitly
  represents the contexts} surrounding the changed statements via the
sub-graphs in {\mvpdg}. It considers the impact of the surrounding
context represented by a context vector on the building of the
embeddings for the code changes. {\tool} uses the contextualized
embeddings to predict if the changes have any vulnerability, and if
yes, it provides assessment grades.

Second, we use the Label, Graph Convolution
Network~\cite{label-gcn} to encode the {\bf program dependencies}
among the entities in the changed code and the ones in the
surrounding un-changed code. This helps overcome the issues
with $n$-grams and capture the statements that might be far
apart but are relevant to the vulnerability.

Third, the detection and assessments of different aspects of a
vulnerability are interdependent on each other. For example, if a
model learns one of the assessment ratings is high (e.g., high
severity or complete unavailability), the vulnerability detection (VD) outcome
must be positive. If the VD outcome is negative, the assessment ratings for
all the aspects must be {\em none}. The assessments for different
aspects could also affect one another, e.g., between the availability and
integrity of a system. Thus, we leverage 
%During the assessment of one aspect, we also consider the impacts of
%the other aspects by leveraging
a {\bf multi-task learning} model to propagate the learning from one
task to another (each task learns to assess one aspect). Vulnerabiity
detection is also a task in the multi-task learning scheme as {\tool} will
provide assessment scores when a vulnerability is detected.


%Experimental results
We have conducted experiments to evaluate {\tool} on real-world
vulnerabilities. Our results on a C vulnerability dataset
show that {\tool} achieves F-score of 25.5\% and MCC of 26.9\%
relatively higher than the state-of-the-art
DeepCVA~\cite{deepCVA-ase21}.  The vulnerability detection result from
{\tool} is also improved over the state-of-the-art ML/DL-based VD
approaches from 13.1-–29.8\% in precision, 15.8-–29.2\% in recall,
and 16.5–-25.9\% in F-score. The results on a Java dataset with 1,229
vulnerabilities show that {\tool} achieves F-score of 20.4\% and MCC
of 28.0\% relatively higher than DeepCVA~\cite{deepCVA-ase21}.

%Our sensitivity analysis shows that all designed components in {\tool}
%contribute positively to its high accuracy.

For the insights, we conducted experiments to show that the
higher~accuracy of {\tool} over DeepCVA roots from~our designed
components.
%We also conducted experiments to show that the higher~accuracy of
%{\tool} over DeepCVA comes from our designed~components.
Our results show that {\em our novel code change embeddings} help
{\tool} better in classifying the commits into the classes for
vulnerability assessment than DeepCVA. Moreover, we used an {\em
  explainable AI tool} to show that {\tool} indeed leverages the key
features in {\em program dependencies} for its correct
assessments.

The key contributions of this work include

{\bf 1. {\tool}: commit-level vulnerability detection and assessment
  model} overcomes the issues in the existing model in $n$-gram
representations of dependencies and contexts.

{\bf 2. Our novel context-aware, graph-based embeddings for code
  changes}
%Our {\em contextualized embeddings for code changes}
integrate dependencies and contexts. This embedding model
is applicable for other down-stream tasks.


{\bf 3. Empirical evaluation.} We evaluated {\tool} 
against the state-of-the-art approach.
Our model/code are available at~\cite{cat-website}.

