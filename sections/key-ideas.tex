\subsection{Key Ideas}
\label{key-ideas:sec}

We develop {\tool}, a Context-aware, Graph-based,
Commit-level~Vulnerability Detection and Assessment Model that detects
any vulnerability in a committed change and provides the CVSS
assessment grades for it if any.  {\tool} is designed with the
following ideas:

\vspace{1pt}
\indent {\bf Key Idea 1 [Multi-Task Learning among Vulnerability Detection
and Assessments]}. From Observation 1, we leverage multi-task learning
to propagate the mutual impact between learning for vulnerability
detection and learning for vulnerability assessment. Moreover, the
learning of the assessment for one aspect could also affect the
learning for another (e.g., between Availability and Access
Complexity). Therefore, our multi-task learning scheme is designed
with one task for VD, and each task for each vulnerability assessment
type (VAT). We use the joint loss function for all the tasks. When the
VD's outcome is positive, {\tool} will provide the assessment
ratings. Otherwise, non-impact scores are given.

%If a model learns one of the assessment ratings is high (e.g.,
%complete unavailability), the VD outcome must be positive. If it is
%negative, the assessment ratings for all the aspects must be
%none. Morever, the learning of one aspect has impact on that of
%another one. Thus, we leverage multi-task learning to propagate such
%mutual learning. We consider the detection is one task of multi-task
%learning as when its outcome is positive, {\tool} will provide the
%assessment. Otherwise, non-impact scores are given.

\vspace{1pt}
\indent {\bf Key Idea 2 [Contextualized Embeddings for
Code Changes with Graph-based Representation Learning]}. To overcome
the limitation of code change representation, we introduce a
graph-based model to build the {\em
contextualized embeddings for code changes} that integrate both {\em
program dependencies} and {\em surrounding contexts}.

Unlike existing code change embedding
approaches~\cite{cc2vec,commit2vec} that code changes are represented
as sequences, we explicitly represent code changes and the
context of a change via a graph representation, called multi-version
PDG~\cite{flexeme-fse20}. The graph consists of
the program entities of both versions before and after the changes,
and their dependencies. The context is defined as the surrounding,
un-changed nodes of the changed node.
%
We use Label, Graph Convolution Network (Label-GCN)~\cite{label-gcn}
to build the contextualized embeddings for code changes, considering
the contexts as the weights in computation.
%
%The context vectors built from a Label, Graph Convolution Network
%(Label-GCN)~\cite{label-gcn} are then used as the weights representing
%the impacts of contexts in building the contextualized embeddings for
%code changes.
We use past vulnerabilities and experts' ratings to train {\tool}
to build such embeddings, and use them for classification.
%
%With such embeddings, we train {\tool} with the past vulnerabilities
%and human experts' ratings in CVSS to predict any vulnerability and
%its assessment.

%We train the Label-GCN~\cite{label-gcn} to learn the embeddings that
%are used in our automated assessment of the vulnerability types.


\vspace{1pt}
\indent {\bf Key Idea 3 [Program Dependencies in Code Change Representation
via Graph-based Neural Network]}.
%Graph-based Neural Network for
%Code Change Representation Learning of Program Dependencies]}.
We integrate program dependencies into {\tool}. We leverage
Label-GCN~\cite{label-gcn} to represent the program dependencies among
the changed code and the surrounding un-changed code.
%
The graph enables a partial order among program entities in a PDG,
rather than enforcing a total order as in $n$-gram.
%
This enables {\tool} overcome the issues with $n$-grams and capture
the distant statements yet relevant to VDA.



%\vspace{1pt}
%\indent {\bf Key Idea 2 [Contextualized Embeddings for Code Changes].}
%We design a context-aware, graph-based, representation learning model
%to {\em learn the contextualized embeddings (vectors) for the code
%  changes} that integrate {\em program dependencies} among the program
%elements, and the {\em contexts} of~code changes. We train a
%Label-GCN~\cite{label-gcn} to learn the embeddings that are used in our
%automated assessment of the vulnerability assessment types (VATs).


