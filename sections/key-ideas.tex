\vspace{-6pt}
\subsection{Key Ideas}
\label{key-ideas:sec}

We develop {\tool}, a Context-aware, Graph-based,
Commit-level~Vulnerability Detection and Assessment Model that
detects any vulnerability in a committed change and provides
the CVSS assessment grades of it if any.
%a vulnerability-introduc\-ing commit and contexts, and outputs
%CVSS assessment grades.
{\tool} is designed with the following ideas:

\vspace{1pt}
%\indent {\bf Key Idea 1 [Graph-based, Representation Learning (RL) Model to learn Contextualized Embeddings for Code Changes]}.
\indent {\bf Key Idea 1 [Contextualized Embeddings for
Code Changes with Graph-based Representation Learning]}. To overcome
the limitation of code change representation, we introduce
%Based on the observations, both contexts and dependencies among the
%code changes are useful in vulnerability assessment prediction. Thus,
%we design our
a graph-based, representation learning model to build the {\em
contextualized embeddings for code changes} that integrate both {\em
program dependencies} and {\em surrounding contexts of the changes}.

Unlike some existing code change embedding
approaches~\cite{commit2vec} that code changes are represented as
sequences, we explicitly represent code changes and the surrounding
context of a change via a graph representation, called multi-version
program dependence graph~\cite{flexeme-fse20}. The graph consists of
the program entities of both versions before and after the changes,
and their dependencies. The context is defined as the surrounding,
un-changed nodes of the changed statement node. The context vectors
built from a Label, Graph Convolution Network
(Label-GCN)~\cite{label-gcn} are then used as the weights representing
the impacts of contexts in building the contextualized embeddings for
code changes. With such embeddings, we train {\tool} with the past
vulnerabilities and human experts' ratings in CVSS to predict any
vulnerability and its assessment.

%We train the Label-GCN~\cite{label-gcn} to learn the embeddings that
%are used in our automated assessment of the vulnerability types.


\vspace{1pt}
\indent {\bf Key Idea 2 [Program Dependencies in Code Change Representation
via Graph-based Neural Network]}.
%Graph-based Neural Network for
%Code Change Representation Learning of Program Dependencies]}.
From Observation~1, we integrate program dependencies into our SV
assessment model. We leverage the Label-GCN~\cite{label-gcn} to
represent the program dependencies among the changed code and the
surrounding un-changed code.
%
The graph enables a partial order among program entities in a PDG,
rather than enforcing a total order as in $n$-gram.
%
This enables {\tool} overcome the issues
with $n$-grams and capture the statements that
%the semantically related statements that
are far apart but are relevant to the assessment.

\vspace{1pt}
\indent {\bf Key Idea 3 [Multi-Task Learning between Vulnerability Detection
and Vulnerability Assessment]}. We leverage multi-task learning to
propagate the learning to assess one aspect to that of another.
We consider vulnerability detection is one task as part of
multi-task learning as when the detection is positive, {\tool}
will provide the assessment. Otherwise, non-impact scores are given.

%\vspace{1pt}
%\indent {\bf Key Idea 2 [Contextualized Embeddings for Code Changes].}
%We design a context-aware, graph-based, representation learning model
%to {\em learn the contextualized embeddings (vectors) for the code
%  changes} that integrate {\em program dependencies} among the program
%elements, and the {\em contexts} of~code changes. We train a
%Label-GCN~\cite{label-gcn} to learn the embeddings that are used in our
%automated assessment of the vulnerability assessment types (VATs).


