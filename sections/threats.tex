\subsection{Threats to Validity and Limitations}
The threats come from the following aspects: (1) \emph{Programming
languages (PLs).}  Our approach has been tested on Java and C
commits. However, the techniques used in {\tool} are not tied to Java
or C.  In principle, our approach can applied to other
PLs. (2) \emph{Generalization of the results.}  Our comparisons with
DeepCVA were only carried out on the publicly available C and Java
datasets. Further comparisons with the baselines on other datasets should be
done.
%To compare with other VD baselines, we did not compare at the commit level.
%snapshot detection.
%(3) \textbf{Training and Tuning.} It is impossible to test the entire hyperparameter space. However, on the Java dataset (used by DeepCVA), we reused the parameters reported in DeepCVA for training and tuning. On the C dataset,

Our approach also has room for further improvements. First, {\tool}
does not work well for the code changes that are common but have
impacts on the far-apart, un-changed parts of the project. Second,
{\tool} fails in the assessment for complex changes that program
dependencies cannot capture, e.g., event-driven programs.  Finally,
the detection component could be improved further with a more dedicated
model on vulnerability detection.
