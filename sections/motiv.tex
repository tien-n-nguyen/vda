\section{Motivation}
\label{motiv:sec}

\subsection{Motivating Example}
\label{exe:sec}

%CVE-2021-37714: description
%CVSS Scores

%Input-Output

%Code Change: src/main/java/org/jsoup/parser/HtmlTreeBuilder.java

%Crash-point or infinite loop: HtmlTreeBuilderState.java

\begin{figure}[t]
  \begin{flushleft}
    \footnotesize
\textbf{Vulnerability Details: CVE-2021-37714}\\
\textbf{1. Description}:
{\em jsoup is a Java library for working with HTML. Those using jsoup versions prior to 1.14.2 to parse untrusted HTML or XML may be vulnerable to DOS attacks. If the parser is run on user supplied input, an attacker may supply content that causes the parser to get stuck (loop indefinitely until cancelled), to complete more slowly than usual, or to throw an unexpected exception. This effect may support a denial of service attack. The issue is patched in version 1.14.2. There are a few available workarounds. Users may rate limit input parsing, limit the size of inputs based on system resources, and/or implement thread watchdogs to cap and timeout parse runtimes.

  Publish Date : 2021-08-18 Last Update Date : 2022-02-07}

\textbf{2. Vulnerability Type(s)}: Denial Of Service

{\bf 3. CVSS Score:} ...\\

{\bf 4. Detailed CVSS Grades:}\\
\end{flushleft}
  \centering
  \tabcolsep 3pt
  \footnotesize
  \begin{tabular}{lll}
   Vulner. Assess. Type   & Value & Description \\
      \hline
    Confidentiality Impact & {\bf None}  & No impact to the confidentiality \\
    Integrity Impact & {\bf None}  & No impact to the integrity \\
    Availability Impact & {\bf Complete} & There is reduced performance or\\
    & & interruptions in availability\\
    Access Complexity & {\bf Low} & Specialized access conditions or \\
    & & extenuating circumstances do not exist\\
    & & Little knowledge is required to exploit\\
    Authentication & {\bf Not Req} & Authentication is not required \\
    & & to exploit the vulnerability\\
    Gained Access & {\bf None}  & No gained access with the vulnerability \\
    Acccess Vector & {\bf Local} & The vulnerability is in the local parser \\
    \end{tabular}%
  \label{CVSS:tab}%
\caption{Vulnerability Details: CVE-2021-37714}
\label{CVSS-tab}
\end{figure}

\begin{figure}[t]
	\centering
	\lstset{
		numbers=left,
		numberstyle= \tiny,
		keywordstyle= \color{blue!70},
		commentstyle= \color{red!50!green!50!blue!50},
		frame=shadowbox,
		rulesepcolor= \color{red!20!green!20!blue!20} ,
		xleftmargin=1.5em,xrightmargin=0em, aboveskip=1em,
		framexleftmargin=1.7em,
                numbersep= 5pt,
		language=Java,
    basicstyle=\scriptsize\ttfamily,
    numberstyle=\scriptsize\ttfamily,
    emphstyle=\bfseries,
                moredelim=**[is][\color{red}]{@}{@},
		escapeinside= {(*@}{@*)}
	}
	\begin{lstlisting}[]
// .../jsoup/parser/HtmlTreeBuilderState.java
boolean process(Token t, HtmlTreeBuilder tb) { ...
  if (t.isCharacter()&& inSorted( (*@{\color{red}{tb.currentElement().normalName()}@*),InTableFoster)){
     ...
     return tb.process(t);
  }
  ...
  } else {
      tb.popStackToClose(name);
(*@{\color{orange}{- \quad \quad tb.resetInsertionMode();}@*)
(*@{\color{orange}{- \quad \quad if (tb.state() == InTable) \{}@*)
(*@{\color{cyan}{+ \quad \quad if (!tb.resetInsertionMode()) \{}@*)
         tb.insert(startTag);
         return true;
      }
(*@{\color{red}{\quad \quad \quad return tb.process(t, InHead);}@*)
      ...
}
	\end{lstlisting}
        \vspace{-15pt}
        \caption{Code Change at Version 1.12.1 for CVE-2021-37714}
        \vspace{-6pt}
        \label{fig:motiv-code}
\end{figure}

%     tb.newPendingTableCharacters();
%     tb.markInsertionMode();
%     tb.transition(InTableText);

%A commit-level, vulnerability assessment tool takes as input a
%committed code change and estimates its impacts on the system under
%development with regards to confidentiality, integrity, availability,
%complexity, severity, etc. That code change might be deemed
%as vulnerable by a commit-level vulnerability detection tool.

Let us present an example from an HTML parser, named {\em jsoup}, and
our observations. Figure~\ref{CVSS-tab} displays the information on
the vulnerability CVE-2021-37714 that was reported on {\em jsoup}, and
published on 08/18/21. The change that was deemed to contribute to the
vulnerability were committed at version 1.12.1 to the method
\code{process(Token,HtmlTreeBuilder)} of the
\code{Html\-Tree\-Builder\-State} class (lines 10--11, and 12 of
Figure~\ref{fig:motiv-code}). That change directly uses the value
returned from \code{reset\-Inser\-tion\-Mode()} as the condition to
insert \code{startTag} (line 13). With this change, certain input HTML
code with a specific start tag could make the program go to line 16
with a recursive call to the method \code{process(...)}. That~call
resulted in an NullPointerException at line 3.
%as noted in the log:~{\em ``java.\-lang.\-Null\-Pointer\-Ex\-ception:
%  Cannot invoke "org.\-jsoup.\-nodes.\-Element.\-normalName()" because
%  the return value of
%  "org.\-jsoup.\-parser.\-HtmlTree\-Builder.\-current\-Element()" is
%  null.''}.
In other cases, the parser can get stuck, i.e., {\em ``loop
  indefinitely until canceled''} as described in the official
description of CVE-2021-37714. Figure~\ref{CVSS-tab} also shows the
Common Vulnerability Scoring System grades (CVSS) given by security
experts for various \underline{v}ulnerability \underline{a}ssessment
\underline{t}ypes (VATs) for that CVE. Due to the above effects, the
availability impact for this vulnerability is rated as {\em Complete}
(i.e., for some inputs, there will be reduced performance and
interruptions in available services).

To build a commit-level vulnerability detection and assessment tool,
one could take advantage of the records on the vulnerabilities and
their assessments by security experts in that CVSS
system~\cite{first-website}. A natural next question is what features
in a code change are useful for such detection and assessment. Toward
answering that, from the above example, we make the following
observations.

\vspace{2pt}
\noindent {\bf Observation 1 [Mutual Impact of Learning for
    Vulnerability Detection and Learning for Vulnerability
    Assessment].} In Figure~\ref{fig:motiv-code}, if a model learns
that the availability assessment of this vulnerability is {\em
  Complete} (i.e., system could be completely unavailable), it could
learn that this change is a vulnerability-introducing one (i.e., the
detection outcome is positive). On the other hand, if a model learns
that this is a vulnerability-introducing change, it could learn that
one of the assessment aspects must be higher than {\em
  None}. Otherwise, if a model decides that this is not a vulnerable
case, all the assessment outcomes must be {\em None}. Unfortunately,
none of existing VD and VA approaches take advantage of this mutual
impact.

%VD: no assessment
%VA: VD -> VA



While the above vulnerability potentially causes damage, both the
detection and assessment for that vulnerability are late. That could
lead to more systems being affected by that vulnerability. Thus, it is
desirable to detect and assess a potential vulnerability as soon as
developers committed their vulnerable code to a repository during
software development. There exist the machine learning (ML) approaches
that automatically analyze the newly committed code, detect software
vulnerabilities~\cite{perl2015vccfinder,zhou2017automated,chen2019large}.
Others provide the assessments at the commit time~\cite{deepCVA-ase21}. For
vulnerability assessment, those approaches have leveraged the manual
assessments from security analysts in CVSS to build labeled data to
train their ML models. However, no approach supports both commit-level
vulnerability detection and assessment at the same time.
%Even though the CVSS system provides manual assessments after the
%vulnerabilities were detected/reported, the grades from security
%analysts actually are helpful as the labeled data for a machine
%learning (ML) to learn to automatically predict the gradings for a
%newly committed code change. In fact, DeepCVA~\cite{deepCVA-ase21}
%uses a $n$-gram representation with a ML model to learn from those
%manually assessments in CVSS to predict for a new commit.
A natural next question is what features in a code change are useful
for such detection and assessment. Toward answering that, from the
above example, we make the following observations.

%developed a machine learning (ML) model that learns from the existing
%grading from security experts to provide the new grading for a
%committed code change that was deemed to be vulnerable.  This type of
%commit-level automated vulnerability assessment together with a
%vulnerability detection (VD) tool are very useful in helping
%developers to early detect and assess the impacts of the detected
%vulnerability as soon as the code is committed. However, the
%state-of-the-art approach for commit-level vulnerability assessment
%is still limited as explained in the following observations.

\vspace{2pt}
\noindent {\bf Observation 1 [Program Dependencies].}  {\em To
  evaluate the impacts w.r.t. different VATs, a model needs to
  consider the program dependencies among the statements}. For
example, to assess Availability, one needs to check the potential
infinite loop or null-pointer exception, and examines the control and
data dependencies between the changed line 12 and the line 16. That is
where the method \code{process} is recursively called, which leads to
the null-pointer exception at line~3 (\code{currentElement()}
returns null). Unfortunately, the state-of-the-art vulnerability
assessment models, e.g., DeepCVA~\cite{deepCVA-ase21} with $n$-grams
($n$=1,3,5)
%to capture the surrounding code. $n$-grams are
are limited in capturing the dependencies in the surrounding code. The
important statements having the dependencies with the changed
statements can help with the assessment, but {\em can be far apart}
(e.g., line 12, line 16, and line 3). They cannot be well captured
with the $n$-grams of limited lengths of 1--5. The tokens within an
$n$-gram distance from the changed code might not be relevant to the
vulnerability. Moreover, $n$-grams require an order in source
code. However, the source code order might not reflect the execution~order. In Figure~\ref{fig:motiv-code}, line 12 has data/control flow to
line 16, which has data/control flow to line 3 via recursion.


%The order should be on the dependency graph, rather than on source
%code.
\vspace{1pt}
\noindent {\bf Observation 2 [Context].} By examining {\em only the tokens
involving in the changes} (e.g., the tokens
\code{tb},~\code{reset\-Insertion\-Mode}, \code{state}, and \code{InTable}
in the deleted lines 10--11, and the inserted line 12), a model can
not decide if the vulnerability could have impact on the system's
availability or not.
%, i.e., deciding if the {\em Availability Impact}
%rating is {\em None}, {\em Partial}, or {\em Complete}.
Generally, {\em the same/similar changes occurring in different
  surrounding contexts might cause different effects, leading to
  different grades for the VATs}. For example, adding a null check:
\code{if p != null} is a common change in many places. However, it
could prevent a null-pointer exception in some context, while does not
in the others.
%(e.g., the availability impact in this case).
Unfortunately, DeepCVA~\cite{deepCVA-ase21} does not capture well the
contexts of the changes. First, it uses $n$-grams ($n$=1,3,5), which
is limited as explained. Second, DeepCVA does not explicitly
distinguish the context (un-changed code) from the changed code, which
confuses the model with two training instances having the same
context+code change, but with different code
changes or contexts. 

%do not explicitly modeling the context

