\begin{figure}[t]
\centering
\lstset{
      		numbers=left,
		numberstyle= \tiny,
		keywordstyle= \color{blue!70},
		commentstyle= \color{red!50!green!50!blue!50},
		frame=shadowbox,
		rulesepcolor= \color{red!20!green!20!blue!20} ,
		xleftmargin=1.5em,xrightmargin=0em, aboveskip=1em,
		framexleftmargin=1.5em,
                numbersep= 5pt,
		language=Java,
                basicstyle=\scriptsize\ttfamily,
                numberstyle=\scriptsize\ttfamily,
                emphstyle=\bfseries,
                moredelim=**[is][\color{red}]{@}{@},
		escapeinside= {(*@}{@*)}
	}
	\begin{lstlisting}[]
private: Status DoCompute(OpKernelContext* ctx) { ...
(*@{\color{purple}{\fbox{+ DatasetBase* finalized\_dataset;}}}@*)
(*@{\color{violet}{\fbox{+ TF\_RETURN\_IF\_ERROR(FinalizeDataset(ctx, dataset, \&finalized\_dataset));}}}@*)
  std::unique_ptr<IteratorBase> iterator;
(*@{\color{black}{- TF\_RETURN\_IF\_ERROR(}@*)(*@{\color{black}{dataset}@*)(*@{\color{black}{->MakeIterator(\&iter\_ctx,/*parent=*/nullptr,.));}}@*)
(*@{\color{violet}{\fbox{+ TF\_RETURN\_IF\_ERROR(finalized\_dataset->MakeIterator(\&iter\_ctx,/*parent=*.));}}}@*)
  std::vector<Tensor> components;
(*@{\color{black}{- components.reserve(}@*)(*@{\color{black}{dataset}@*)(*@{\color{black}{->output\_dtypes().size());}}@*)
(*@{\color{red}{\fbox{+ components.reserve(finalized\_dataset->output\_dtypes().size());}}}@*) ...
}
\end{lstlisting}
\vspace{-15pt}
\caption{Contributions of different statements in correct
vulnerability prediction and classification of VATs by {\tool}}
\vspace{-6pt}
\label{gnn-example}
\end{figure}

%/*parent=*/nullptr,.));

%{\color{red}{ CVSS Score -> 4.6

%$Severity$ -> Medium

%GNNExplainer Sub-graph (6 Nodes)-> Line 5, 6, 10, 12, 14, one line not shown in the example.}}

\noindent {\bf Example.} Figure~\ref{gnn-example} shows an example of
the vulnerability-introduc\-ing change for CVE-2021-37650. The change
introduced the variable \code{finalized\_dataset} at line 2,
representing a dataset that was populated at line 3 and used at lines
6 and 9. However, the input was not validated, and the code at line 9
assumed only string inputs and interpreted numbers as valid strings.
When computing the CRC of the record, this resulted in~heap buffer
overflow. {\tool} correctly predicted this vulnerability and its
assessment {\em Severity=Medium}. GNNExplainer pointed out that
{\tool} used the changed statements and their dependencies at lines 2, 3, 6, 9
and another line (not shown), to perform classification. This is
correct since despite that the line 9 is the fixed line (later), lines 2, 3, 6
are parts of the control/data flows leading to line 9. We see that
{\tool} used the vulnerability-relevant statements and
dependencies in correct prediction.

%Thus, program dependencies encoded in our embeddings are the
%key features enabling {\tool} in correct classification.


%Tien
%Thus, {\tool} correctly used the statements/dependencies as the key
%features in its assessment. That is, our graph representations with
%program dependencies encoded by the embeddings have positive
%contributions to {\tool}'s accuracy.


%Code assumes only strings inputs and then interprets numbers as valid
%`tstring`s. Then, when trying to compute the CRC of the record this
%results in heap buffer overflow.


%private:
%Status DoCompute(OpKernelContext* ctx) {
%	...
%	IteratorContext iter_ctx(std::move(params));
%(*@{\color{cyan}{+ \quad	DatasetBase* finalized\_dataset;}}@*)
%(*@{\color{cyan}{+ \quad	TF\_RETURN\_IF\_ERROR(FinalizeDataset(ctx, dataset, \&finalized\_dataset));}}@*)
	
%	std::unique_ptr<IteratorBase> iterator;
%(*@{\color{red}{-	\quad   TF\_RETURN\_IF\_ERROR(dataset->MakeIterator(\&iter\_ctx, /*parent=*/nullptr, "ToTFRecordOpIterator", \&iterator));}}@*)
%(*@{\color{cyan}{+	\quad   TF\_RETURN\_IF\_ERROR(finalized\_dataset->MakeIterator(\&iter\_ctx, /*parent=*/nullptr, "ToTFRecordOpIterator", \&iterator));}}@*)
	
%	std::vector<Tensor> components;
%(*@{\color{red}{- \quad	components.reserve(dataset->output\_dtypes().size());}}@*)
%(*@{\color{cyan}{+ \quad	components.reserve(finalized\_dataset->output\_dtypes().size());}}@*)
%	bool end_of_sequence;
%	...
%}
