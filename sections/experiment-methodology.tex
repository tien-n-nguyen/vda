\subsection{Experimental Methodology}
\label{method:sec}

\subsubsection{\textbf{Datasets}}\label{dataset}
%\begin{table}[h]
%	\caption{Statistics on Dataset}
%	\vspace{-15pt}
%	\begin{center}
%		\renewcommand{\arraystretch}{1}
%		\begin{tabular}{p{1.6cm}<{\centering}p{3cm}<{\centering}p{1cm}<{\centering}p{1.5cm}<{\centering}p{2cm}<{\centering}}
%		\end{tabular}
%		\label{DataSet}
%	\end{center}
%	\vspace{-10pt}
%\end{table}

We used two vulnerability datasets in C and Java: BigVul
2.0~\cite{bigvul-msr20} and CVAD~\cite{deepCVA-ase21}
(Table~\ref{dataset:tab}).
%Both were built by crawling the Common Vulnerabili\-ties and Exposures
%(CVE) database and the code repositories mentioned in the CVEs.
Both were manually~checked and used in prior
research~\cite{bigvul-msr20,li2021vulnerability,deepCVA-ase21}. Experiments were
conducted on a server with 16 core CPU and a single Nvidia A100 GPU.

%Table~\ref{dataset:tab} shows the statistics of the datasets.

%In this paper, we evaluate approaches on vulnerabilities in C and Java using two different datasets: BigVul~2.0~\cite{} and DeepCVA-dataset~\cite{} (CVAD).  Both BigVul2.0 and DeepCVA-dataset are built by crawling the public Common Vulnerabilities and Exposures database~\cite{} and CVE-related source code repositories. BigVul~2.0 is in C and DeepCVA-dataset is in Java. Table~\ref{} shows the statistics of two datasets.


\begin{table}[t]
	\caption{Statistics of BigVul and CVAD Datasets}
        \tabcolsep 2.5pt
        \vspace{-10pt}
	\begin{center}
%        \footnotesize
\small
		\renewcommand{\arraystretch}{1}
		\begin{tabular}{l|p{1.5cm}<{\centering}|p{2cm}<{\centering}}
			
        	Datasets	& BigVul (C) & CVAD (Java)\\\hline
		\# of Projects  & 303 & 246 \\\hline
	    \# of Vulnerabilities & 3336 & 542\\\hline
     	\# of Vulnerability Introducing Commits & 7851 & 1229\\\hline

		\end{tabular}
		\label{dataset:tab}
	\end{center}
%	\vspace{-10pt}
\end{table}







%To evaluate the performance of automated commit-level vulnerability assessment, we use the following two metrics: F1-Score and Matthews Correlation Coefficient (MCC). These two metrics have been widely, commonly, and frequently in the literature (check deepcva page 6, evaluation metrics for literature). Due to the fact that our {\tool} is built for multi-class classification, we used the macro F1-Score and multi-class version of MCC as in~\cite{le2021deepcva}.


\subsubsection{\textbf{Analysis Approaches for RQs\\}}

\noindent\textbf{RQ1. Comparison on Vulnerability Detection on C dataset}

\emph{Baselines}. We include {\bf VCCFinder}~\cite{perl2015vccfinder}
(a commit-level ML-based VD), and other ML VD tools:
\textbf{VulDeePecker} \cite{li2018vuldeepecker}, \textbf{Devign}
\cite{zhou2019devign}, \textbf{SySeVR} \cite{li2021sysevr},
\textbf{Russell} {\em et al.}  \cite{russell2018automated},
\textbf{Reveal} \cite{chakraborty2021deep}, and {\bf IVDetect}
\cite{li2021vulnerability}. Except VCCFinder, we ran the others
on the code after commit. 

\emph{Procedure.}  We use all vulnerable methods and randomly select
the same number of non-vulnerable methods from the fixed version
projects, to build a dataset with the vul:non-vul ratio of 1:1. We
also evaluate the tools with the real-world ratio of 9:1. We randomly split the
data 80\%, 10\%, 10\% on the project basis without changing the
vul:non-vul ratio for training, tuning, and testing.

\emph{Parameter Tuning.}  For {\tool}, we used
autoML~\cite{NNI} for tuning the following hyper-parameters to have the
best performance: (1) Epoch size (100, 200, 300); (2) Batch size (64,
128, 256); (3) Learning rate (0.001, 0.003, 0.005, 0.010); (4) word embeddings length (150, 200, 250, 300). We
tuned DeepCVA's parameters from its documentation.
%We use AutoML~\cite{NNI} on all models to automatically tune
%hyper-parameters on the tuning dataset. We tuned \tool with the
%parameters \textit{batch size, hidden size, learning rate, dropout
%  rate}. The hyper-parameters we tuned for the baselines can be found
%in their papers. We choose the hyper-parameters with the best accuracy
%for each model.

\emph{Evaluation Metrics.} We use the following metrics to measure the
effectiveness of a model: $Recall = \frac{TP}{TP+FN}$, $Precision =
\frac{TP}{TP+FP}$, and $F\-score =
\frac{2*Recall*Precision}{Recall+Precision}$.
%\begin{equation}
%	Recall = \frac{TP}{TP+FN}
%\end{equation}
%\begin{equation}
%	Precision = \frac{TP}{TP+FP}
%\end{equation}
%\begin{equation}
%	F\_score = \frac{2*Recall*Precision}{Recall+Precision}
%\end{equation}
where TP = True Positives; FP = False Positives; FN = False Negatives; TN = True Negatives.

%Recall measures how many of the vulnerable methods can be correctly detected, while Precision is used to measure how many of the detected vulnerable methods are indeed labeled as vulnerable in the collected dataset. F-score is used to reflect the overall performance by combining both the Recall and Precision.

%\vspace{3pt}
\noindent\textbf{RQ2. Comparison on Vulnerability Assessment on C dataset.}

{\em Baseline and Procedure.} We compare {\tool} with
DeepCVA~\cite{deepCVA-ase21}. We ran the models on BigVul. We used the
same longitudinal setting in~\cite{deepCVA-ase21,falessi2020need} for
training, validation, and testing to mimic the real-world scenario in
which the older vulnerabilities are used for training to assess the
newer. Specifically, we sorted all the commits in a chronological
order based on their absolute time. We divided the commits into 10
equal folds from oldest to newest. For a fold $k$ ($k$ $\le$ $8$), we used
all the folds 1 to $k$ for training, the $(k$+$1)^{th}$ and $(k$+$2)^{th}$
folds for validation and testing. Models are tuned as in~RQ1.

%\subsubsection{\textbf{Evaluation Metrics}}

%To evaluate the models' performance in automated vulnerability
%assessment,

\emph{Evaluation Metrics.} We use the same metrics in
DeepCVA~\cite{deepCVA-ase21}: F-score and Matthews Correlation
Coefficient (MCC). F-score ranges from 0 to 1 (the best), and MCC
ranges from -1 to 1 (the best). F-score is a metric to evaluate
classification tasks. We chose F-score to handle the class imbalance
prevalent in some of the VATs. Because we evaluate the classification
models with multiple classes, we used the macro
F-Score~\cite{spanos2018multi} and the multi-class version of
MCC~\cite{gorodkin04}. The overall MCC is computed as the average of
the MCCs for all classification tasks as in DeepCVA~\cite{deepCVA-ase21}.

%We turned {\tool} with the following key hyper-parameters ....
%We turned DeepCVA with the following key hyper-parameters ...
%We use macro F1-Score and multi-class MCC to evaluate both approaches.


%We also tuned the models as in RQ1.

%In this RQ, we compare {\tool} with the same baseline in RQ1, DeepCVA. We ran {\tool} and DeepCVA on the Java dataset, DeepCVA-dataset~\cite{} (CVAD).  We used the same time-series data splitting strategy as the one in RQ1 to train, validate, and test both models.


%we randomly split the BigVul into xx\% for training, xx\% for validation, and xx\% for testing.


%We turned {\tool} with the following key hyper-parameters .... As we reused the dataset from the DeepCVA, we directly tune DeepCVA with the parameters reported in their paper. Same as RQ1, we use macro F1-Score and multi-class MCC to evaluate both approaches.

%\vspace{1pt}
\noindent\textbf{RQ3. Code Change Embedding Analysis.} We evaluate the impact of the embeddings in the classification for VATs.

\noindent \textbf{RQ4. Program Dependencies.} We use
Explainable AI to show that {\tool} in fact leverages program
dependencies in VDA.

%\vspace{1pt}
%\noindent\textbf{RQ5. Overlapping Analysis.}  We counted the
%vulnerabilities that were correctly assessed by {\tool} but
%not by DeepCVA and vice versa, and the ones
%that were correctly classified by both.


\vspace{1pt}
\noindent\textbf{RQ5. Ablation Study on Multi-Task Learning and
  Context.}  We evaluate the impacts of key factors in {\tool}: (1)
multi-task learning, and (2) change context. We conducted ablation analysis
by removing a main factor from the model and made a comparison.
%d it with the xcomplete model.

\vspace{3pt}
\noindent\textbf{RQ6. Comparison on Assessment on CVAD (Java dataset).}

{\em Baseline and Procedure.} We compared {\tool} with the baseline
DeepCVA~\cite{deepCVA-ase21}. We used the same procedure, tuning, and
longitudinal setting as in RQ1. We used the Java~dataset.

%For (3), we used GNNExplainer~\cite{GNNExplainer} (Section~\ref{discuss:sec}).



%We aim to show the insights that {\tool}'s embeddings for code changes
%help improve automated vulnerability assessment over
%DeepCVA~\cite{deepCVA-ase21}.



%We aim to study the projections of the code changes in the commits
%with respect to different vulnerability assessment types.

