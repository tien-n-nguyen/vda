\section{Conclusion}

This paper proposes {\tool}, a Context-aware, Graph-based,
Commit-level Vulnerability Detection and Assessment Model that
evaluates a commit, detects any vulnerability and provides the CVSS
assessment grades.
%for the detected vulnerability.
The key advances in {\tool} over the existing approaches include 1) multi-task learning between vulnerability
detection and assessments of different aspects, 2) code change
embedding model that integrates program dependencies and contexts, and
3) graph-based representations of dependencies and
contexts. Our evaluation shows that on a vulnerability dataset in C,
CAT achieves F-score of 25.5\% and MCC of 26.9\% relatively higher
than the baseline in vulnerability assessment. In Java dataset, CAT
achieves F-score of 31\% and MCC of 33.3\% relatively higher. Our
contextualized embeddings have other applications related to code
changes.

%(1) Integrating program dependencies into our commit-level
%vulnerability assessment approach; (2) Learning contexualized
%embeddings for code changes that integrating program dependencies
%among the program elements and the contexts of code changes.



%We propose a new context-aware, graph-based, commit-level vulnerability assessment approach, namely {\tool} to improve the most state-of-the-art commit-level vulnerability assessment approach, DeepCVA. The key ideas that enable our approach are (1) Integrating program dependencies into our commit-level vulnerability assessment approach; (2) Learning contexualized embeddings for code changes that integrating program dependencies among the program elements and the contexts of code changes. In our {\tool}, we leverage and train the Label, Graph Convolution Network (Label-GCN) to learn the program dependencies among the changed code and its surrounding unchanged code to generate the contextualized embeddings for code changes in the automated assessment of the vulnerability assessment types.

%We have conducted extensive empirical studies to evaluate {\tool} on two different datasets in C and Java. {\tool} is able to outperform the most recent and state-of-the-art automated vulnerability assessment approach, DeepCVA. Particularly, on the C dataset, {\tool} can improve DeepCVA by xx\% and xx\% in terms of F1-score and MCC, respectively. On the Java dataset, {\tool} can improve DeepCVA by xx\% and xx\% in terms of F1-score and MCC, respectively. Furthermore, our sensitivity analysis shows that all designed components of {\tool} can positively contribute to the model.





 %Designing a context-aware, graph-based, representation learning model to learn the contextualized embeddings for the code changes.








