\section{Related Work}

%Here, we summarize some studies relevant to our study.



%{\tool} is closely related to DeepCVA as explained in
%Section~\ref{intro:sec}.

%The most relevant work to {\tool} is the state-of-the-art commit-level approach, DeepCVA, working on code changes. However, {\tool} is designed to overcome the shortcomings of DeepCVA in code change representation and embeddings, program dependencies and surrounding contexts of changed code. Our empirical studies have shown that {\tool} can outperform DeepCVA.

%\vspace{3pt}
%\noindent{\bf ML-based Vulnerability Prediction. }

\subsubsection*{\bf ML-based Vulnerability Prediction}
%Commit-level vulnerability detection is an important direction.
Machine Learning has been applied in commit-level vulnerability
detection~\cite{perl2015vccfinder,zhou2017automated,chen2019large}.
%developed commit-level VD models that leveraged ML models.
VCCFinder~\cite{perl2015vccfinder} trains a SVM classifier to flag
suspicious commits. We used only code change features for our
experiment. Zhou and Sarma~\cite{zhou2017automated}'s works on commit
messages and bug reports. It uses an ensemble model to combine
multiple classifiers.
%with random forest,
%gaussian naive bayes, k-nearest neighbors, SVM, etc.
%Many studies (e.g, \cite{li2021vulnerability,
%  zhou2019devign,li2021vuldeelocator,li2020automated,chakraborty2021deep,hin2022linevd})
%also built machine or deep learning models to detect vulnerabilities
%in source code.
{\tool} supports both vulnerability detection and
assessment.

Deep learning (DL) has been applied
to detect
vulnerabilities~\cite{li2021vulnerability,zhou2019devign,li2021vuldeelocator,li2020automated,chakraborty2021deep,hin2022linevd,scandariato2014predicting,neuhaus2007predicting,shin2010evaluating,neuhaus2009beauty,yamaguchi2012generalized,yamaguchi2011vulnerability}.
%For example, some approaches train a DL model on different code
%representations to detect vulnerabilities, such as the lexical
%representations of functions in a synthetic
%codebase~\cite{harer2018learning}, code snippets related to API calls
%to detect two types of vulnerabilities~\cite{li2018vuldeepecker},
%syntax-based, semantics-based, and vector
%representations~\cite{li2018sysevr}, graph-based
%representations~\cite{zhou2019devign}.
Harer {\em et al.}~\cite{harer2018learning} leverages RNN model. Lin
{\em et al.}~\cite{lin2017poster} learns function repreentations via
AST for VD. Russell {\em et al.}~\cite{russell2018automated} combine
the neural features of functions with random forest as
a classifier.
%It does not consider program dependencies.
Harer {\em et al.}~\cite{harer2018automated} compare the
effectiveness in VD of using source code and the compiled
code. VulDeePecker~\cite{li2018vuldeepecker} uses a RNN trained on
program slices from API calls for VD. SySeVR~\cite{li2021sysevr}
expands VulDeePecker by including the program slices from 
syntactic units. Devign~\cite{zhou2019devign} uses Gated Graph
Recurrent Layers on program
graphs. Reveal~\cite{chakraborty2020deep} uses CPG with
GGNN. IVDetect~\cite{li2021vulnerability} focuses on interpretation
and directly uses PDG with GCN. LineVul~\cite{linevul-msr22} use
BigVul dataset to train a transformer-based model which has over 150K
training instances. To avoid under-training of LineVul and an unfair
comparison (given that it has over 110M parameters), we chose to not
compare with it.
%Assessment is the key for earlier prioritization and resource
%relocation for identified vulnerabilities. Furthermore, detection
%models can be used to help detect vulnerabilities and then {\tool}
%assesses them.

%\noindent{\bf Code Embedding Learning.}

%\vspace{2pt}
%\noindent{\bf Automated Vulnerability Assessment. }

\vspace{-2pt}
\subsubsection*{\bf Automated Vulnerability Assessment}

Distinct software vulnerabilities can have different levels of threats
and severity, and require
assessment~\cite{nayak2014some,le2019automated,khan2018review}.
%Thus, it is desired to assess vulnerabilities for prioritizing actions
%and resources
%so that more severe ones can be studied and patched
%before more exploits~\cite{khan2018review}.
The automated approaches have been recently
proposed~\cite{bozorgi2010beyond,allodi2014comparing,deepCVA-ase21}.
Bozorgi {\em et al.}~\cite{bozorgi2010beyond} propose a SVM-based
approach to predict whether a vulnerability will be exploited or not.
%Specifically, they extracted over 93k features from vulnerability data
%and classify vulnerabilities using these features.
%This work represents the early effort to replace the small-scale
%rating-based assessment framework by learning-based approaches.
Lamkanfi {\em et al.}~\cite{lamkanfi2010predicting} predict the
severity of a reported bug using text mining algorithms on bug reports.
%to analyze the textual descriptions on the bugs.
%
Han {\em et al.}~\cite{han2017learning} propose a multi-class text
classification DL-based model that is based on the
description to predict the severity level of a vulnerability.
%critical, high, medium, and low in CVSS~\cite{first-website}.

%Specifically, given a vulnerability description, they classified the
%text into one of the severity levels, for example, critical, high,
%medium, low of the Common Vulnerability Scoring System (CVSS)
%framework~\cite{first-website}.
Georgios {\em et al.}~\cite{spanos2018multi} adopt a multi-target
classification coupled with text analysis on vulnerability
descriptions to predict their characteristics and scores.
Le {\em et al.}~\cite{le2019automated} propose a ML-based approach to
learn the word features in vulnerability description,
and handle the extended concepts in the description.
%Unlike the above studies built for analyzing
%vulnerability description, some
Other studies~\cite{ponta2018beyond,ponta2020detection} leverage code
patterns in fixing commits of third-party libraries to assess
vulnerabilities. In comparison,
%{\tool} is fundamentally different from the above studies, as it
{\tool} supports commit-level vulnerability detection and assessment
using code changes.


%\noindent{\bf Code Change Embeddings.}

\vspace{-4pt}
\subsubsection*{\bf Code Change Embeddings}
Our work is also related to code change embedding
models~\cite{cc2vec,commit2vec}. Those approaches mainly treat code as
sequences and do not consider structures and/or program
dependencies. The key departure points of {\tool} include the use of
graph representation to model the changes and dependencies, as well as
the surrounding context to build the embeddings.
