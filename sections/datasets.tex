\subsubsection{\textbf{Datasets}}\label{dataset}
%\begin{table}[h]
%	\caption{Statistics on Dataset}
%	\vspace{-15pt}
%	\begin{center}
%		\renewcommand{\arraystretch}{1}
%		\begin{tabular}{p{1.6cm}<{\centering}p{3cm}<{\centering}p{1cm}<{\centering}p{1.5cm}<{\centering}p{2cm}<{\centering}}
%		\end{tabular}
%		\label{DataSet}
%	\end{center}
%	\vspace{-10pt}
%\end{table}

We used two vulnerability datasets in C and Java: BigVul
2.0~\cite{bigvul-msr20} and CVAD~\cite{deepCVA-ase21}
(Table~\ref{dataset:tab}).
%Both were built by crawling the Common Vulnerabili\-ties and Exposures
%(CVE) database and the code repositories mentioned in the CVEs.
Both were well-established and used in prior
research~\cite{bigvul-msr20,fse21,deepCVA-ase21}. Experiments were
conducted on a server with 16 core CPU and a single Nvidia A100 GPU.

%Table~\ref{dataset:tab} shows the statistics of the datasets.

%In this paper, we evaluate approaches on vulnerabilities in C and Java using two different datasets: BigVul~2.0~\cite{} and DeepCVA-dataset~\cite{} (CVAD).  Both BigVul2.0 and DeepCVA-dataset are built by crawling the public Common Vulnerabilities and Exposures database~\cite{} and CVE-related source code repositories. BigVul~2.0 is in C and DeepCVA-dataset is in Java. Table~\ref{} shows the statistics of two datasets.


\begin{table}[t]
	\caption{Statistics of BigVul and CVAD Datasets}
        \tabcolsep 2.5pt
        \vspace{-10pt}
	\begin{center}
%        \footnotesize
\small
		\renewcommand{\arraystretch}{1}
		\begin{tabular}{l|p{1.5cm}<{\centering}|p{2cm}<{\centering}}
			
        	Datasets	& BigVul (C) & CVAD (Java)\\\hline
		\# of Projects  & 303 & 246 \\\hline
	    \# of Vulnerabilities & 3336 & 542\\\hline
     	\# of Vulnerability Introducing Commits & 7851 & 1229\\\hline

		\end{tabular}
		\label{dataset:tab}
	\end{center}
%	\vspace{-10pt}
\end{table}





