\subsubsection{{\bf Comparison on Vulnerability Detection on C Dataset (RQ1)}}
\label{sec:vd-result}

\begin{table} [t]
  \caption{Vulnerability Detection on C Dataset (RQ1)}
  \vspace{-10pt}
  \begin{center}
    \small
			\begin{tabular}{cccc}
				\toprule
\textbf{Approach} & \textbf{Precision} & \textbf{Recall} & \textbf{F-score}  \\
\midrule
 VCCFinder~\cite{perl2015vccfinder}      & 0.28             & 0.13          & 0.18          \\
VulDeePecker~\cite{li2018vuldeepecker}      & 0.55             & 0.77          & 0.64           \\
SySeVR~\cite{li2021sysevr}            & 0.54             & 0.74           & 0.63           \\
Russell \textit{et al.}~\cite{russell2018automated} & 0.54       & 0.72          & 0.62            \\
Devign~\cite{zhou2019devign}           & 0.56             & 0.73          & 0.63           \\
Reveal~\cite{chakraborty2021deep}            & 0.62             & 0.69          & 0.65      \\
IVDetect~\cite{li2021vulnerability} & 0.54              & 0.77    & 0.67           \\
\midrule
{\tool}            & {\bf 0.72} & {\bf 0.90} & {\bf 0.81}\\
                \bottomrule
			\end{tabular}
			\label{tab:rq1}
                        \vspace{-12pt}
		\end{center}
\end{table}

%DeepVD            & \textbf{\begin{tabular}[c]{@{}c@{}}0.6997\\ (↑12.53\%)\end{tabular}}
%                  & \textbf{\begin{tabular}[c]{@{}c@{}}0.8827\\ (↑0.32\%)\end{tabular}}
%                  & \textbf{\begin{tabular}[c]{@{}c@{}}0.7806\\ (↑19.54\%)\end{tabular}} \\

As seen in Table~\ref{tab:rq1}, to detect vulnerability, {\tool}
improves over all the baselines in all the metrics. Specifically,
{\tool} relatively improves over the baseline models from {\bf 13.1\%--
  157\%} in Precision, from {\bf 15.8\%--592\%} in Recall, and from
{\bf 16.5\%--350\%} in F-score.


%To understand how {\tool} is complementary to the baselines, we
%compute how the top-100 result from {\tool} overlap with that from the
%top-performing baseline, IVDetect~\cite{fse21}. As seen in
%Figure~\ref{RQ1-overlap}, {\tool} can detect {\bf 19} vulnerable
%methods that IVDetect missed, while IVDetect can detect only {\bf 10}
%that {\tool} missed. Both detected the same 25 vulnerabilities. This
%result shows that {\tool} not only detects more vulnerabilities than
%the baselines, but also complements them.

%\begin{figure}[t]
%	\centering
%	\includegraphics[width=1.3in]{overlap.png}
%        \vspace{-5pt}
%	\caption{Overlapping Results between {\tool} and IVDetect}
%	\label{RQ1-overlap}
%\end{figure}

%-----------------------------------------------------------------

%\begin{figure}[t]
%	\centering
%	\lstset{
%		numbers=left,
%		numberstyle= \tiny,
%		keywordstyle= \color{blue!70},
%		commentstyle= \color{red!50!green!50!blue!50},
%		frame=shadowbox,
%		rulesepcolor= \color{red!20!green!20!blue!20} ,
%		xleftmargin=1.5em,xrightmargin=0em, aboveskip=1em,
%		framexleftmargin=1.5em,
%               numbersep= 5pt,
%		language=C++,
%    basicstyle=\scriptsize\ttfamily,
%    numberstyle=\scriptsize\ttfamily,
%    emphstyle=\bfseries,
%    moredelim=**[is][\color{red}]{@}{@},
%	escapeinside= {(*@}{@*)}
%	}
%\begin{lstlisting}[]
%BIO *PKCS7_dataDecode(PKCS7 *p7, EVP_PKEY *pkey, BIO *in_bio, X509 *pcert) {
%  ...
%  if (evp_cipher != NULL) {
%    ...
%    if (pcert == NULL) {
%      for (i = 0; i < sk_PKCS7_RECIP_INFO_num(rsk); i++) {
%        ri = sk_PKCS7_RECIP_INFO_value(rsk, i);
%        (*@{\color{red}{if (pkcs7\_decrypt\_rinfo(\&ek, \&eklen, ri, pkey) < 0)}}@*)
%           goto err;
%        ERR_clear_error();
%      }
%   } else {...}
%}
%\end{lstlisting}
%        \vspace{-12pt}
%        \caption{CVE-2019-1563: A vulnerable code in OpenSSL with 186 lines of code (after removed comments and empty lines). ReVeal and IVDetect failed to detect this but \tool did.}
%        \label{fig:rq1_example}
%\end{figure}


%PDT    & 145 & 144  \\
%EFG    & 145 & 295  \\
%PDG    & 145 & 477  \\
%CPG    & 622 & 1393 \\

%Examining the results,
{\tool} performs much better than commit-level
VCCFinder~\cite{perl2015vccfinder} because VCCFinder uses traditional
SVM. {\tool} also often performed better than the snapshot baselines,
Reveal~\cite{chakraborty2021deep} and
IVDetect~\cite{li2021vulnerability}, in the cases that the code
changes are relevant to the vulnerability. For example, a statement
(e.g., null check) was removed, leading to the vulnerable code (e.g.,
Null-Pointer Exception). Because examining the changes, {\tool} can
detect better those vulnerability-introducing changes. The baselines
work only on the new version.

In Section~\ref{sec:ai}, we used {\em Explainable AI to display the
changed statements} in the code that contributes to the detected
vulnerability (Figure~\ref{gnn-example}). This is another advantage
from commit-level VD (pointing to the {\em fine-grained}
vulnerability-introducing change) that the VD tools working on the new
version only do not have.


%in the code with complex PDGs (which IVDetect uses as a key feature
%for detection) and CPGs (which Reveal uses for detection). Moreover,
%{\tool} detects vulnerabilities well with its code change embeddings.

%improper exception/error-handling.

%Fig.~\ref{fig:rq1_example} shows the vulnerable code from OpenSSL that
%was reported in CVE-2019-1563. The code (186 lines of code) has the
%PDG with 145 nodes and 477 edges, and the CPG with 622 nodes and 1,393
%edges. In contrast, PDT+EFG has 145 nodes and 295 edges (including the
%error-handling flows from line 8 to lines 9 and 10). Thus, complex PDG
%or CPG produce much noise for IVDetect and Reveal, which missed this
%vulnerability. {\tool} with its PDT+EFG features detects~well this
%type of vulnerability with improper error-handling (e.g., the
%incorrect code at line 8, which was fixed with an additional
%condition).

%=======================================

We also use a real-world vulnerability setting with a 9:1
non-vulnerability to vulnerability ratio. We randomly selected 10\% of
the vulnerable instances in the test set ten times, and finally took
the average of the F-scores. We report that the F-scores for
{\tool} and the top baseline IVDetect are 45.3\% and 25.4\%,
respectively. {\tool} exhibits a consistent trend with IVDetect while
outperforming it by 78.1\%. F-score is lower than in Table~\ref{tab:rq1} due to
unbalanced data.

%We report that F-score for {\tool} and the top baseline IVDetect are
%(45.3\%, 23.3\%) and (25.4\%, 20.4\%) respectively. {\tool} exhibits
%consistent trends with IVDetect while outperforming it by 78.1\% and
%14\% in Accuracy and F-score.
