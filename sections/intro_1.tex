\section{Introduction}
\label{intro:sec}

%The need for cyber resilience is increasingly important in our~techno\-logy-dependent society, where software systems have been, and will continue to be the target of cyber attackers.

Software Vulnerabilities (SVs) are security weaknesses and flaws that one can exploit in cyber-attacks.
Such vulnerable code can be found and reported by scanning a version of a software project with the help of vulnerability detection tools~\cite{li2018vuldeepecker,zhou2019devign,li2021sysevr}.
%Vulnerability detection (VD) tools~\cite{li2018vuldeepecker,zhou2019devign,li2021sysevr} scan a version of a software project and report the presence of any vulnerable code.
However, it is crucial to identify SVs as
%early as possible because late corrections could cost much more and even cause severe damage due to more affected systems.
soon as possible because late fixes and consequent damage due to affected systems are severe.
Toward that, {\em commit-level vulnerability
  detection} (VD)
approaches~\cite{perl2015vccfinder,zhou2017automated,chen2019large}
have been proposed to catch SVs as soon as new code changes are
committed to the code repositories during
% software development. They are used to warn developers early on potential vulnerable code and also help better prioritize the code review task.
the software development process. They are useful in warning
developers early and isolating vulnerability-inducing, fine-grained code change.
%about potentially vulnerable code patterns in their code change.

%which will also help better prioritize the code review tasks.

%New software security vulnerabilities are discovered on an almost
%daily basis. It is crucial to identify software vulnerabilities as
%early as possible, because late corrections of errors could cost much
%more than early correction or even cause more severe
%damage. Recognizing this, researchers propose the approaches to detect
%software vulnerabilities as soon as new code changes are committed to
%the code repositories during software development. Those approaches
%are referred to as {\em commit-level vulnerability detection
%  (VD)}~\cite{perl2015vccfinder,zhou2017automated,chen2019large}. They
%could be used to warn developers early on potential vulnerable code.

While addressing software vulnerabilities is crucial, providing developers with the assessment on the impacts of the detected vulnerability in the system under development is of equal interest.
%For potential vulnerable code, it would be equally important to provide developers with the assessment on the impacts of the detected vulnerability on the system under development.
Such knowledge will help better prioritize code review tasks, while showing which areas need patching first.
The assessment could be on the severity of the attack and the levels of damage regarding confidentiality, integrity, availability, etc. However, the commit-level VD approaches~\cite{perl2015vccfinder,zhou2017automated,chen2019large} do not support such automated assessment.
%Recognizing that importance, the commit-level software vulnerability assessment (VA) tools~\cite{deepCVA-ase21} have been proposed to assess a committed code change. Researchers have leveraged the Common Vulnerability Scoring System (CVSS) ratings~\cite{first-website} manually assessed by security analysts for the vulnerabilities as labeled data to train a machine learning (ML) model and applied it on the code change to predict the assessment ratings.
Recognizing this, the state-of-the-art approaches~\cite{deepCVA-ase21} proposed commit-level software vulnerability assessment (VA) tools to assess a committed code change. They leveraged the Common Vulnerability Scoring System (CVSS)~\cite{first-website} ratings (i.e., numerical scores which can be transformed into qualitative representations such as low, medium, high, and critical) manually provided by security analysts to build a learning-based approach to predict assessment ratings for a code change possessing vulnerabilities.
%In CVSS, the security experts provide the numerical scores that can be transformed into qualitative representations (such as low, medium, high, and critical) for vulnerability assessment.
However, the existing VA tools still have limitations.

%First, the vulnerability assessment models~\cite{deepCVA-ase21} work only on a committed change that was detected as vulnerable by a VD tool.
First, they work only on a committed code change that has already been identified as vulnerable by a VD tool.
This strategy~that runs VD and then VA (i.e., VD $\rightarrow$ VA)
%would suffer the cascading error.
is prone to cascading errors.
%For the false positive cases from VD
For instance, in commits that have wrongly been
identified~as~vulnerable by a VD tool, the VA tool would give
incorrect assessments, which should not be given at all.  Also, the VA
tool fails to predict~assessment patterns in commits that are
vulnerable, but missed by the VD tool. As a result, the VD
$\rightarrow$ VA strategy is limited, and would~not take advantage of
the {\em mutual benefits of the learning of detection on the learning
  of assessment, and vice versa}.  Our philosophy is driven by this
inter-dependence, wherein the former could help isolate the part of
the code change that is vulnerability-inducing, which, in turn, could
help assess it better. Specifically, if a model learns that one of the
assessment ratings is high (e.g., high severity or complete
unavailability), the vulnerability detection outcome must be
positive. If the detection outcome is negative, the assessment ratings
for all the aspects must be at the lowest (e.g., none). Thus, it is
more beneficial for a model to jointly learn to detect and assess
vulnerabilities at~the same time, leading to performance enhancement
in both tasks. Besides, an alternative solution with an existing VD
tool running on the snapshot after a commit would not be able to
isolate the vulnerability-inducing code change.

Second, the state-of-the-art VA approaches~\cite{deepCVA-ase21} are
inept in their {\em representation of code changes},
%and do not take into account the {\em context} of such changes.
which are not sufficiently {\em context-aware}.
%They use $n$-gram
They learn an $n$-gram
%representation for code changes that are insufficient to capture the
%changed statements useful for the VD/VA but far apart.
representation for code changes that captures limited local context,
and is incapable of learning from VD/VA-relevant program statements
that are further apart from the changed statements.  As a result, the
statements in the $n$-gram representation might not all contain
important features for VD/VA. Such a sequential representation
learning also enforces an order in source code, which might not be the
order of execution (e.g., in the cases of loop, branch condition, or
recursion), which contributes to detecting vulnerabilities involving
execution and exception flows. Moreover, given that a vulnerability
attack exploits the control and data flows, a VD/VA tool needs to
consider {\em program dependencies}.  This facet is ignored
in the existing VA tools.
 %With $n$-grams, the VA tools do not model the program
%dependencies. Finally, it does not represent well the {\em code
%  context} surrounding a change. The same change occurring in
%different contexts might cause different effects, leading to different
%impact grades.
Finally, the context of un-changed code is indistinguishable and not
represented separately from the changes. As a result, it can be
confused by two training instances that have the same combination
of changes and contexts overall, but with different changes and
associated contexts.

In this work, we present {\tool}, a Context-aware, Graph-based,
Commit-level Vulnerability Detection and Assessment model that
evaluates a changed code to detect the presence of a vulnerability,
while also providing corresponding CVSS assessment scores for the
detected vulnerability. We build {\tool} via the novel integration of
three key ideas.  First, we {\em integrate vulnerability detection and
  assessment (VDA)} processes such that our tool can directly be~built
into a repository to evaluate committed code changes and provide
just-in-time assistance. Due to the inter-dependence of these
processes, we adopt a {\em multi-task learning} approach to propagate
the learning between VD and VA, leading to better performance of both
tasks. Moreover, the assessments for different aspects could also
affect one another~\cite{deepCVA-ase21}, e.g., between the
availability and integrity of a system. Thus, our multi-task learning
scheme includes the VD task and the assessment tasks for different
security aspects.

Second, we address the limitation of code change representations described earlier via
%Second, regarding the code change representation, we develop
a novel
{\em context-aware, graph-based, representation learning model}
%to {\bf learn the contextualized embeddings for the code changes}
that
integrates the {\em program dependencies}, and the surrounding {\em
  contexts} of code changes. We leverage both versions of the program
dependence graphs (PDGs), i.e., before and after a commit via the multi-version {\mvpdg}~\cite{flexeme-fse20}. To build such embeddings
for code changes, {\tool} {\em explicitly represents the contexts}
surrounding the changed statements via the sub-graphs in {\mvpdg}. It
considers the impact of the context represented by a
context~vector on the building of the embeddings for the code
changes. It uses the contextualized embeddings to predict if the
changes have any vulnerability, and if yes, to provide assessment
grades. Third, it utilizes the Label, Graph Convolution
Network~\cite{label-gcn} to encode the {\em program dependencies}
among the entities in the changed code and the ones in the surrounding
un-changed code. This helps overcome the issues with $n$-grams and
capture the statements that might be far apart, but still relevant to the
vulnerability.

%---------- OLD INTRO ----------------
%%%For potential vulnerable code, it would be also useful to provide
%%%developers with the assessment on the impacts of the vulnerability on
%%%the system under development. In the Common Vulnerability Scoring
%%%System (CVSS)~\cite{first-website}, the security experts and analysts
%%%have manually provided the assessments in terms of numerical ratings
%%%to quantify different aspects of a software vulnerability including
%%%the exploitability, the impacts and the severity of the attacks, the
%%%level of damages of the vulnerabilities,
%%%etc. (Figure~\ref{CVSS-tab}). The numerical scores can be transformed
%%%into qualitative representations (such as low, medium, high, and
%%%critical) to help organizations properly assess and prioritize their
%%%vulnerability management processes~\cite{first-website}. Despite its
%%%usefulness, there are restrictions. First, human efforts are needed
%%%for the {\em manual assessment}, leading to the delays in the process.
%%%Second, CVSS scores are provided {\em after} a vulnerability was
%%%reported. Thus, for an early action, an {\em automated, vulnerability
%%%  assessment} is desired at the committing time. The {\em
%%%  vulnerability detection and assessment tools (VDA)} can be
%%%integrated into the repositories as the code is checked in to provide
%%%just-in-time assistance.
%%%
%%%Recognizing such importance, the state-of-the-art, commit-level
%%%software vulnerability assessment tools~\cite{deepCVA-ase21} have been
%%%proposed to assess a committed code change. Researchers have leveraged
%%%the aforementioned CVSS ratings manually assessed by security analysts
%%%for the vulnerabilities as labeled data to train a machine learning
%%%(ML) model and applied it on the code change to predict the assessment
%%%ratings. However, none of the existing commit-level vulnerability
%%%assessment tools supports {\em both commit-level vulnerability
%%%  detection and assessment}. Importantly, they are still limited in
%%%the {\em representation of code changes} and do not take into account
%%%the {\em context} of such changes.
%%%
%%%Specifically, they are limited in using $n$-grams in representing code
%%%changes. $n$-grams are insufficient to capture the execution flows
%%%among code elements surrounding the changes. {\em The important
%%%  statements having dependencies with the changed statements can be
%%%  useful for the assessment, but can be far apart}. For example, many
%%%Denial of Service (DoS) vulnerabilities are related to null-pointer
%%%exceptions, exception flows, segmentation faults, etc. The execution
%%%flows among statements are not well-represented via $n$-grams with
%%%such limited lengths. Moreover, {\em the statements in an $n$-gram
%%%  might be irrelevant to the current vulnerability}. Thus, the
%%%assessment of a DoS vulnerability with $n$-grams could be
%%%imprecise. {\em $n$-grams also enforce an order in source code}, which
%%%might not be the execution order, e.g., in the cases of loop,
%%%condition, or recursion. The execution order and dependencies among
%%%the statements are important in an DoS. Thus, to assess the impact of
%%%a vulnerability, $n$-gram is limited.
%%%
%%%Second, to estimate the impacts of software vulnerability, a model
%%%needs to consider {\bf program dependencies} among statements since an
%%%attack to a vulnerability involves the~exploits of the control and
%%%data dependencies. The state-of-the-art vulnerability assessment tools
%%%capture code changes as code tokens in short sequences with $n$-grams,
%%%thus, does not model the program dependencies. Finally, it
%%%does not represent well the {\bf code context} surrounding a
%%%change. The same change occurring in different contexts might cause
%%%different effects, leading to different impact grades. In the
%%%state-of-the-art approaches~\cite{deepCVA-ase21}, the context of
%%%un-changed code is not~distinguishable and not represented
%%%separately from the code changes. The model can be confused by the two
%%%training instances having the same combination of code changes and
%%%contexts, but with different code changes and contexts
%%%themselves. Those limitations lead to low accuracy in the
%%%existing approaches.
%%%
%%%We present {\tool}, a Context-aware, Graph-based, Commit-level
%%%Vulnerability Detection and Assessment Model to evaluate a changed
%%%code, detect any vulnerability and provide the CVSS
%%%assessment grades for the detected vulnerability. To build {\tool}, we
%%%provide a novel integration among three key ideas. First, to make
%%%prediction on the CVSS assessment grades on code changes, we develop a
%%%novel {\bf context-aware, graph-based, representation learning
%%%  model} to {\bf learn the contextualized embeddings for the code
%%%  changes} that integrate {\em program dependencies}, and the
%%%surrounding {\em contexts} of code changes.  Both versions of the
%%%program dependence graphs (PDGs) before and after the commit are
%%%modeled via the multi-version {\mvpdg}~\cite{flexeme-fse20}.
%%%To build such embeddings for code changes, {\tool} {\em explicitly
%%%  represents the contexts} surrounding the changed statements via the
%%%sub-graphs in {\mvpdg}. It considers the impact of the surrounding
%%%context represented by a context vector on the building of the
%%%embeddings for the code changes. {\tool} uses the contextualized
%%%embeddings to predict if the changes have any vulnerability, and if
%%%yes, it provides assessment grades.
%%%
%%%Second, we use the Label, Graph Convolution
%%%Network~\cite{label-gcn} to encode the {\bf program dependencies}
%%%among the entities in the changed code and the ones in the
%%%surrounding un-changed code. This helps overcome the issues
%%%with $n$-grams and capture the statements that might be far
%%%apart but are relevant to the vulnerability.
%%%
%%%Third, the detection and assessments of different aspects of a
%%%vulnerability are interdependent on each other. For example, if a
%%%model learns one of the assessment ratings is high (e.g., high
%%%severity or complete unavailability), the vulnerability detection (VD)
%%%outcome must be positive. If the VD outcome is negative, the
%%%assessment ratings for all the aspects must be {\em none}. The
%%%assessments for different aspects could also affect one another, e.g.,
%%%between the availability and integrity of a system. Thus, we leverage
%%%a {\bf multi-task learning} model to propagate the learning from one
%%%task to another (each task learns to assess one aspect). Vulnerability
%%%detection is also a task in the multi-task learning scheme as {\tool}
%%%will provide assessment scores when a vulnerability is detected.

%-------------------------------------------------------------------------

%Experimental results
We have conducted experiments to evaluate {\tool} on real-world
vulnerabilities. Our results on a C vulnerability dataset show that
{\tool} achieves 25.5\% and 26.9\% relatively higher than the
state-of-the-art VA tool DeepCVA~\cite{deepCVA-ase21} in F-score and
MCC, respectively. The vulnerability detection result from {\tool} is
improved over the state-of-the-art ML/DL-based VD approaches from
11.3-–146\% in precision, 10.4-–553\% in recall, and 13.4–-322\% in
F-score.
%approaches from 13.1-–29.8\% in precision, 15.8-–29.2\% in recall,
%and 16.5–-25.9\% in F-score.
The results on a Java dataset with 1,229 vulnerabilities show that
{\tool} also achieves 31.0\% and 33.3\% relatively higher
than DeepCVA~\cite{deepCVA-ase21} in F-score and MCC, respectively.

%Our sensitivity analysis shows that all designed components in {\tool}
%contribute positively to its high accuracy.

To gain better insights, we conducted experiments to show that the better
performance of {\tool} roots from our designed components.
%We also conducted experiments to show that the higher~accuracy of
%{\tool} over DeepCVA comes from our designed~components.
Our results show that {\em our novel code change embeddings} help
{\tool} learn better class-separation, i.e., better in classifying the
commits into the classes for vulnerability detection and
assessment. Moreover, we used {\em explainable AI} to show
that {\tool} indeed leverages the correct features in {\em program
  dependencies} among statements for its assessments.
The key contributions of this work include

{\bf 1. {\tool}: commit-level vulnerability detection and assessment
  model (VDA)} that performs VD and VA in tandem, leveraging multi-task
learning to improve both detection and assessment.

%overcomes the issues in the existing model in $n$-gram representations
%of dependencies and contexts.

{\bf 2. Our novel context-aware, graph-based embeddings for code
  changes}
%Our {\em contextualized embeddings for code changes}
integrate program dependencies and contexts. This embedding model
is applicable for other down-stream tasks.


{\bf 3. Empirical evaluation.} We evaluated {\tool}
against the state-of-the-art approach.
Our model/code are available at~\cite{cat-website}.

